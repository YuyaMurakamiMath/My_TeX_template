\documentclass[11pt,a4paper,oneside,lualatex]{ltjsarticle} % LuaLaTeXの場合
%\documentclass[11pt,a4paper,oneside,titlepage,lualatex]{ltjsreport} % 表紙付き, 章から始まる形式

%SumatraPDFの逆順検索でエラーが出た時は以下のコマンドラインをSumatraPDFの設定→オプションで入力する
%"C:\Program Files (x86)\TeXstudio\texstudio.exe" "%f" -line %l

\usepackage{luatexja} % ltjclasses, ltjsclasses を使うときはこの行不要
\usepackage[marginparwidth=0pt,left=25mm,right=25mm,top=30mm,bottom=30mm]{geometry} % 余白の設定
% スマホやタブレットでも読みやすいB5サイズの文書を書くときは, 1行目の\documentclassのオプションで「a4paper」を「b5paper」にして, 余白設定はmargin=10truemmにすると自分好み

% --------------------------------------------------------------------------
%		パッケージとコマンド
% --------------------------------------------------------------------------

\usepackage{mypackage} % よく使うパッケージ. 「C:\w32tex\share\texmf-local\tex\(好きなファイル名)」に置いたmypackage.styを読み込む
\usepackage{mycommand} % 自分で定義したコマンド. 「C:\w32tex\share\texmf-local\tex\(好きなファイル名)」に置いたmycommand.styを読み込む

%\usetikzlibrary{graphs,graphs.standard,graphdrawing} % TikZでグラフを描く
%\usegdlibrary{trees,force,layered} %graphdrawingの子ライブラリ

\usetikzlibrary{knots}
\usetikzlibrary{positioning}

\usepackage{pxrubrica} % \ruby{漢字}{かんじ}のようにしてルビを振る

\usetikzlibrary{decorations.pathmorphing}
\newcommand{\longrightsquigarrow}{\begin{tikzcd}[cramped,sep=scriptsize,ampersand replacement=\&]{}\arrow[r, squiggly]\&{}\end{tikzcd}}

% --------------------------------------------------------------------------
%		ハイパーリンク
% --------------------------------------------------------------------------

%目次にもハイパーリンクが付く. プリアンブルのできるだけ後ろに書く. 
\usepackage[luatex, pdfencoding=auto,hypertexnames=false]{hyperref}
\hypersetup{% hyperrefオプションリスト
	colorlinks=true,
	linkcolor=DarkGoldenrod, % リンクの色
	citecolor=SlateBlue, % 引用文献の色
	urlcolor=violet, % URLの色
}

% --------------------------------------------------------------------------
%		定理環境と相互参照
% --------------------------------------------------------------------------

% 参照番号の設定
\numberwithin{equation}{section} % 式番号
\newtheorem{theoremcounter}{}[section] % 定理番号のオプションを選択. [section]を[chapter]にすれば章番号から始まり「定理 1.1.1」のようになる.
\newtheorem{exercisecounter}{}[] % 演習問題番号のオプションを選択.

% 自作スタイルファイル読み込み
\usepackage{mytheorem} % cleverefパッケージによる定理環境と相互参照. 「C:\w32tex\share\texmf-local\tex\(好きなファイル名)」に置いたmytheorem.styを読み込む. \usepackage{hyperref}の後に書く. 

\usepackage{myprogram} % ハイライト付きソースコード. 「C:\w32tex\share\texmf-local\tex\(好きなファイル名)」に置いたmyprogram.styを読み込む

% --------------------------------------------------------------------------
\begin{document}
% --------------------------------------------------------------------------

\title{TeXノウハウ}
\author{村上友哉}
\date{\today}
\maketitle


% --------------------------------------------------------------------------

\section*{はじめに}

% --------------------------------------------------------------------------

この文書は, 私村上の私的なTeXノウハウをまとめたものです. 

\tableofcontents
%\addcontentsline{toc}{chapter}{目次}

% --------------------------------------------------------------------------

\section{細かいノウハウ} \label{sec:know-how}

% --------------------------------------------------------------------------



% --------------------------------------------------------------------------

\subsubsection{一番左で等号を揃えた式を出す方法}

% --------------------------------------------------------------------------

ここでは\verb|\phant|で\verb|\phantom{{}={}}|が出力されるようにしている。

\href{https://qiita.com/zkou/items/21a2229bb900fd87e760#-%E3%82%92%E5%90%AB%E3%82%80%E4%BE%8B}{[LaTeX] phantomでいい感じに揃える}
によると\verb|\phantom{=}|ではなく\verb|\phantom{{}={}}|と入力することでより良い出力結果が得られるそうです。



% --------------------------------------------------------------------------

\subsubsection{長い$ \rightsquigarrow $を出す方法}

% --------------------------------------------------------------------------

多くの矢印のコマンドは、前に「long」を付け足すことで長い矢印を出力できます。
例えば「\verb|\mapsto|」によって「$ \mapsto $」が出力され、「\verb|\longmapsto|」によって「$ \longmapsto $」が出力されます。

しかしながら、「\verb|\rightsquigarrow|」によって出力される「$ \rightsquigarrow $」には長いバージョンのコマンドが用意されておらず、「\verb|\longrightsquigarrow|」と打ってもエラーになってしまいます。
これを解決するにはプリアンブルに
\begin{verbatim}	
	\usepackage{tikz}
	\usetikzlibrary{cd, decorations.pathmorphing}
	\newcommand{\longrightsquigarrow}{\begin{tikzcd}[cramped,sep=scriptsize,
	ampersand replacement=\&]{}\arrow[r, squiggly]\&{}\end{tikzcd}}
\end{verbatim}
と書けば良いです。
このようにすると「\verb|\longrightsquigarrow|」で$ \longrightsquigarrow $が出力されるようになります。

これは何をやっているかというと、可換図式を描くのに用いられるtikz-cdパッケージを用いて矢印の長さを調整することで「\verb|\longrightsquigarrow|」というコマンドを新しく定義しています。
これは\href{https://tex.stackexchange.com/questions/99017/long-squiggly-arrows-in-latex}{Stack Exchangeの質問``Long Squiggly Arrows in LaTeX''}の回答を参考にしています。
そこでの回答ではxyパッケージを用いて対処していますが、私が利用しているLuaLaTeXではxyパッケージを用いるとコンパイルが通りません。
\href{https://tex.stackexchange.com/questions/328602/lualatex-and-xypic}{Stack Exchangeの質問``LuaLaTeX and xypic''}にあるように、プリアンプルに「\verb|\RequirePackage{luatex85}|」と書いてLuaTeXのバージョンを下げれば対処できますが、あまりバージョンを下げることはしたくありません。
また可換図式を描く用途としてはxyパッケージよりもtikz-cdパッケージの方が後発で使い勝手が良いように思われるので、なるべくtikz-cdパッケージを使いたいという気持ちがあります。
そこで上ではtikz-cdパッケージを利用して書きました。
矢印自体は「\verb|{}\arrow[r, squiggly]&{}|」と書けば良いのですが、技術的な注意点が色々あります。
一つ目は矢印\verb|\arrow[r, squiggly]|の前後を\verb|{}|で挟んでいることで、これは矢印が結ぶ両端をダミーの空白にしています。
二つ目は\verb|&|をそのまま書くとエラーが出てしまうことです。
これは\href{https://tex.stackexchange.com/questions/15093/single-ampersand-used-with-wrong-catcode-error-using-tikz-matrix-in-beamer}{Stack Exchangeの質問``Single ampersand used with wrong catcode" error using tikz matrix in beamer''}にあるようにオプションで\verb|&|の代わりに\verb|\&|を使うように指定すればOKです。
これが\verb|ampersand replacement=\&|の部分です。
また、デフォルトだと前後の空きが大きくなってしまうのを\verb|cramped|で抑制し、矢印の長さを\verb|sep=scriptsize|で調節しています(この部分は\href{https://abenori.blogspot.com/2015/07/tikz-cd.html}{にっき♪:LaTeXで可換図式:tikz-cd}の「インライン」の項を参考にしました)。
矢印の長さは\href{https://ctan.math.washington.edu/tex-archive/graphics/pgf/contrib/tikz-cd/tikz-cd-doc.pdf}[Tikz-cdのドキュメント]の6ページにあるように\cref{tab:tikz_arrow_length}にある6通りのオプションで指定できるが、ここでは最も\verb|\longrightarrow|($ \longrightarrow $)に近い長さの\verb|scriptsize|を採用しました。

\begin{table}[htb]
	\centering
	\setlength{\extrarowheight}{1mm} % 行送りを1mm増やす
	%\rule[0mm]{0mm}{10mm}は行の底を+10mm
	\begin{tabular}{cccccc}
		tiny & small & scriptsize & normal & large & huge \\
		0.45 em & 0.9 em & 1.35 em & 1.8 em & 2.7 em & 3.6 em
	\end{tabular}
	\caption{}
	\label{tab:tikz_arrow_length}
\end{table}


% --------------------------------------------------------------------------

\subsubsection{枠付き定理環境のラベルを通常と同じコードで書く方法}

% --------------------------------------------------------------------------

tcolorboxを用いると定理環境に枠を付けられるが、コードの書き方が通常と異なるため、枠無し定理環境の文書と枠付き定理環境の文書のコードに互換性が無くなってしまう。
これは中々不便であるが、\href{https://tex.stackexchange.com/questions/633490/environments-with-tcolorbox-referencing}{Stack Exchangeの質問``Environments with tcolorbox, referencing''}に解決策が載っている。

% --------------------------------------------------------------------------

\subsubsection{ありとあらゆる数式を書く方法}

% --------------------------------------------------------------------------

LuaLaTeX/XeLaTeXでunicode-mathパッケージを使う。

% --------------------------------------------------------------------------

\section{TeX以外のノウハウ} \label{sec:know-how_others}

% --------------------------------------------------------------------------


% --------------------------------------------------------------------------

\subsubsection{正規表現}

% --------------------------------------------------------------------------


% --------------------------------------------------------------------------

\subsubsection{arXivのファイルをTeX形式でダウンロードする方法}

% --------------------------------------------------------------------------

右上の「Access Paper:」欄にある「TeX Source」をクリックするとarXiv-0000.0000v1.tar.gzというような名前のファイルがダウンロードされるので回答すれば良い。
解凍するには、Windowsの場合コマンドプロンプトで「tar -xvzf arXiv-0000.0000v1.tar.gz」とすればよい。
なおダウンロードフォルダではうまくいかないようなので一度別のフォルダに移動させる必要がある。

% --------------------------------------------------------------------------

\section{} \label{sec:}

% --------------------------------------------------------------------------

% --------------------------------------------------------------------------

\subsubsection{}

% --------------------------------------------------------------------------


% --------------------------------------------------------------------------

\subsubsection{}

% --------------------------------------------------------------------------


% --------------------------------------------------------------------------

\subsubsection{}

% --------------------------------------------------------------------------




\bibliographystyle{alpha}
\bibliography{myrefs_alg_num}

% --------------------------------------------------------------------------
\end{document}
% --------------------------------------------------------------------------