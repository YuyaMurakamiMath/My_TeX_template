\documentclass[11pt,a4paper,oneside,lualatex]{ltjsarticle} % LuaLaTeXの場合
%\documentclass[11pt,a4paper,oneside,titlepage,lualatex]{ltjsreport} % 表紙付き, 章から始まる形式

%SumatraPDFの逆順検索でエラーが出た時は以下のコマンドラインをSumatraPDFの設定→オプションで入力する
%"C:\Program Files (x86)\TeXstudio\texstudio.exe" "%f" -line %l

\usepackage{luatexja} % ltjclasses, ltjsclasses を使うときはこの行不要
\usepackage[marginparwidth=0pt,margin=25truemm]{geometry} % 余白の設定
% スマホやタブレットでも読みやすいB5サイズの文書を書くときは, 1行目の\documentclassのオプションで「a4paper」を「b5paper」にして, 余白設定はmargin=10truemmにすると自分好み

% --------------------------------------------------------------------------
%		パッケージとコマンド
% --------------------------------------------------------------------------

\usepackage{mypackage} % よく使うパッケージ. 「C:\w32tex\share\texmf-local\tex\(好きなファイル名)」に置いたmypackage.styを読み込む
\usepackage{mycommand} % 自分で定義したコマンド. 「C:\w32tex\share\texmf-local\tex\(好きなファイル名)」に置いたmycommand.styを読み込む

% --------------------------------------------------------------------------
%		ハイパーリンク
% --------------------------------------------------------------------------

%目次にもハイパーリンクが付く. プリアンブルのできるだけ後ろに書く. 
\usepackage[luatex, pdfencoding=auto,hypertexnames=false]{hyperref}
\hypersetup{% hyperrefオプションリスト
	colorlinks=true,
	linkcolor=orange, % リンクの色
	citecolor=cyan, % 引用文献の色
	urlcolor=violet, % URLの色
}

% --------------------------------------------------------------------------
%		定理環境と相互参照
% --------------------------------------------------------------------------

%	パッケージ
\usepackage[most]{tcolorbox} % 定理環境に枠を付けるために必要
\tcbuselibrary{breakable, skins, theorems, listingsutf8} % 定理環境に枠を付けるために必要

%枠付き節見出しのために必要
\usetikzlibrary{calc}
\usepackage[explicit]{titlesec}

% 参照番号の設定
\numberwithin{equation}{section} % 式番号を「(1.1)」のように出力
\newtcbtheorem[number within = section]{theoremcounter}{}{}{} % 定理番号のオプションを選択. [number within = section]を[number within = chapter]にすれば章番号から始まり「定理 1.1.1」のようになる.
\newtcbtheorem[]{exercisecounter}{}{}{} % 演習問題番号のオプションを選択. []を[number within = section], [number within = chapter]にすればそれぞれ節, 章番号から始まり「演習問題 1.1」「演習問題 1.1.1」のようになる. [theoremcounter]にすれば定理番号と共通の番号.

% 自作スタイルファイル読み込み
\usepackage{mytheoremcolor} % カラフルな定理環境. 「C:\w32tex\share\texmf-local\tex\(好きなファイル名)」に置いたmytheoremcolor.styを読み込む
\usepackage{myprogram} % ハイライト付きソースコード. 「C:\w32tex\share\texmf-local\tex\(好きなファイル名)」に置いたmyprogram.styを読み込む

% --------------------------------------------------------------------------
%		タイトルのデザイン
% --------------------------------------------------------------------------

\makeatletter
\def\thickhrulefill{\leavevmode \leaders \hrule height 1pt\hfill \kern \z@}
\renewcommand{\maketitle}{\begin{titlepage}%
		\let\footnotesize\small
		\let\footnoterule\relax
		\parindent \z@
		\reset@font
		\null\vfil
		\begin{center}
			\Huge \@title
		\end{center}
		\par
		\hrule height 4pt
		\par
		\vskip 40\p@
		\begin{flushright}
			\LARGE \@author \par
		\end{flushright}
		\vskip 30\p@
		%		\vfil\null
		\begin{flushright}
			{\large \@date}%
		\end{flushright}
		\vskip 30\p@		
		%目次部分
		\begin{contentsbox}
			\makeatletter
			\@starttoc{toc}
			\makeatother
		\end{contentsbox}
	\end{titlepage}%
	\setcounter{footnote}{0}%
}
\makeatother

% --------------------------------------------------------------------------
\begin{document}
% --------------------------------------------------------------------------

\title{カラフルな記事テンプレート}
\author{村上友哉}
\date{\today}

\maketitle

%シンプルな目次
%\tableofcontents


このテンプレートではtcolorboxパッケージというものを用いてカラフルな文書を作成する.

% --------------------------------------------------------------------------

\section{カラフルな定理環境} \label{sec:thm}

% --------------------------------------------------------------------------

カラフルな定理環境をはthm環境を使って

\begin{thm}{見出し}{label_name}
	定理の内容
\end{thm}

として\cref{thm:label_name}で引用する.
定理番号の形式を変更したいときはプリアンブルから設定する.
他にもprop環境で命題, cor環境で系, conj環境で予想, que環境で疑問, dfn環境で定義, nota環境で記号, lem環境で補題, rem環境で注意, clm環境で主張, ex環境で例, obs環境で観察, exc環境で演習問題を出力できる.

\begin{dfn}{}{1}
	
\end{dfn}

\begin{lem}{}{1}
	
\end{lem}

\begin{exc}{}{}
	演習問題だけはデフォルトで番号付けが定理番号とは別の通し番号になっている.
	変更したいときはプリアンブルから設定する.
\end{exc}

% --------------------------------------------------------------------------

\section{各種設定} \label{sec:setting}

% --------------------------------------------------------------------------

\begin{itemize}
	\item  タイトルのデザインはプリアンブルの「タイトルのデザイン」の項目から変更できる. 
	\item 定理環境, セクションの区切り, 目次のデザインはmytheoremcolor.styで設定を変更できる. 
\end{itemize}

なお, これらを弄るのは結構大変である. 
tcolorboxパッケージを勉強しつつ弄ってみるのが楽しいと思う.
tcolorboxパッケージは公式ドキュメントも見栄えが楽しいのでオススメである.

% --------------------------------------------------------------------------

%bibtexで文献引用
\bibliographystyle{alpha}
\bibliography{myrefs} % myrefs.bibに書いた文献データを引用

% --------------------------------------------------------------------------
\end{document}
% --------------------------------------------------------------------------