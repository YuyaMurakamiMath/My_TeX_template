\documentclass[12pt,a4paper,twoside,titlepage,lualatex]{ltjsreport}

%SumatraPDFの逆順検索でエラーが出た時は以下のコマンドラインをSumatraPDFの設定→オプションで入力する
%"C:\Program Files (x86)\TeXstudio\texstudio.exe" "%f" -line %l

\usepackage{luatexja} % ltjclasses, ltjsclasses を使うときはこの行不要
\usepackage[marginparwidth=0pt,margin=25truemm]{geometry} % 余白の設定
% スマホやタブレットでも読みやすいB5サイズの文書を書くときは, 1行目の\documentclassのオプションで「a4paper」を「b5paper」にして, 余白設定はmargin=10truemmにすると自分好み

%\renewcommand{\bibname}{参考文献}
%\renewcommand{\proofname}{\bf 証明}
%\renewcommand{\labelenumi}{{\rm(\arabic{enumi})}}

% --------------------------------------------------------------------------
%		パッケージとコマンド
% --------------------------------------------------------------------------

\usepackage{mypackage} % よく使うパッケージ. 「C:\w32tex\share\texmf-local\tex\(好きなファイル名)」に置いたmypackage.styを読み込む
\usepackage{mycommand} % 自分で定義したコマンド. 「C:\w32tex\share\texmf-local\tex\(好きなファイル名)」に置いたmycommand.styを読み込む

% --------------------------------------------------------------------------
%		ハイパーリンク
% --------------------------------------------------------------------------

%目次にもハイパーリンクが付く. プリアンブルのできるだけ後ろに書く. 
\usepackage[luatex, pdfencoding=auto,hypertexnames=false]{hyperref}
% ハイパーリンクの色
%\hypersetup{% hyperrefオプションリスト
%	colorlinks=true,
%	linkcolor=DarkGoldenrod, % リンクの色
%	citecolor=SlateBlue, % 引用文献の色
%	urlcolor=violet, % URLの色
%}
\hypersetup{colorlinks=false} % ハイパーリンクの色付けを無効化

% --------------------------------------------------------------------------
%		定理環境と相互参照
% --------------------------------------------------------------------------

% 参照番号の設定
\numberwithin{equation}{chapter} % 式番号を「(1.1)」のように出力
\newtheorem{theoremcounter}{}[chapter] % 定理番号のオプションを選択. [section]を[chapter]にすれば章番号から始まり「定理 1.1.1」のようになる.
\newtheorem{exercisecounter}{}[] % 演習問題番号のオプションを選択. []を[section], [chapter]にすればそれぞれ節, 章番号から始まり「演習問題 1.1」「演習問題 1.1.1」のようになる. [theoremcounter]にすれば定理番号と共通の番号.

% 自作スタイルファイル読み込み
\usepackage{mytheorem} % cleverefパッケージによる定理環境と相互参照. 「C:\w32tex\share\texmf-local\tex\(好きなファイル名)」に置いたmytheorem.styを読み込む. \usepackage{hyperref}の後に書く. 

\usepackage{myprogram} % ハイライト付きソースコード. 「C:\w32tex\share\texmf-local\tex\(好きなファイル名)」に置いたmyprogram.styを読み込む

% --------------------------------------------------------------------------
\begin{document}
% --------------------------------------------------------------------------

\title{
{\Large 東北大学大学院理学研究科 数学専攻 \\[1cm]
博士前期課程 修士論文} \\[3cm]
\huge ここに修論タイトルを書く \\[5cm]
}
\author{B8SM1031 村上友哉}
\date{\Large 2020年 1月 } 
\maketitle
\thispagestyle{empty}
\clearpage

%シンプルな目次
\tableofcontents
\thispagestyle{empty} % 目次はページ番号なし
\newpage % 目次だけで1ページ使う
\setcounter{page}{1} % 本文を1ページ目から始める
\renewcommand{\thepage}{\arabic{page}} % ページ番号をアラビア数字に指定

% --------------------------------------------------------------------------

\chapter{序論} \label{chap:intro}

% --------------------------------------------------------------------------

本論文では~~について考察する...

動機は...

本論文で解決したい問題は...

先行研究で行われていることを紹介する...

しかし~~の場合には十分な進展が得られていない.
これは~~という困難があるためである...

本論文の主定理は...

\begin{thm}
	主定理を書く
\end{thm}

この結果を得る上で, 上述した困難を~~によって乗り越えた...

今後の課題を挙げる... 

本論文の構成を述べる.
\cref{chap:prelim}では...


\newpage

% --------------------------------------------------------------------------

\section*{謝辞} \label{sec:acknowledgement}

% --------------------------------------------------------------------------


\newpage

% --------------------------------------------------------------------------

\section*{記号} \label{sec:notation}

% --------------------------------------------------------------------------

本論文を通して用いる記号を以下に述べる.

\begin{itemize}
	\item 本論文では, 環といえば単位元を持つ可換環とする. 
	\item $ \bbH \coloneq \{ \tau = x+y\sqrt{-1} \mid x, y \in \R, y>0 \} $を複素上半平面とする. 
\end{itemize}

% --------------------------------------------------------------------------

\chapter{準備} \label{chap:prelim}

% --------------------------------------------------------------------------

本章では基礎事項の準備を行う. 

% --------------------------------------------------------------------------

\section{二次形式} \label{sec:quad_form}

% --------------------------------------------------------------------------

この節では二次形式の基礎事項を述べる... 

\begin{lem}\label{lem:quadsubsp}
	$ R $を環, $ L $を有限生成自由$ R $加群, $ Q \colon L \to R $を二次形式, 
	$ M \subset L $を部分自由$R$加群であって$ \rank_R L = \rank_RM $なるものとすると, 
	$ \det M = d^2 \cdot \det L $を満たす$ d \in R $が存在する. 
\end{lem}

\begin{proof}
	$ L, M $の基底に関する$ Q $の行列表示を取ればよい. 
\end{proof}

\cref{lem:quadsubsp}より...

% --------------------------------------------------------------------------$

\bibliographystyle{plain}
\bibliography{myrefs}

% --------------------------------------------------------------------------
\end{document}
% --------------------------------------------------------------------------